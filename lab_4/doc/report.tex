\documentclass{article}

\usepackage[english, russian]{babel}
\usepackage{geometry}
\usepackage{graphicx}
\usepackage{listings}
\usepackage{xcolor}
\usepackage[14pt]{extsizes}
\usepackage{amsmath}
\usepackage{setspace}
\usepackage{multirow}
\usepackage{tocloft}
\usepackage{indentfirst} 
\usepackage{lipsum}
\usepackage{caption}
\usepackage{cmap}
\usepackage[utf8]{inputenc}
\usepackage[T2A]{fontenc}

\captionsetup[figure]{name={Рисунок},labelsep=endash}
\captionsetup[table]{singlelinecheck=false, labelsep=endash}

\renewcommand{\cftsecleader}{\cftdotfill{\cftdotsep}}
\geometry{pdftex, left = 3cm, right = 1cm	, top = 2cm, bottom = 2cm}
\onehalfspacing

\setlength{\parindent}{1,25cm}
\lstdefinestyle{python}{
	language={Python},
	basicstyle=\footnotesize\ttfamily,
	frame=single,
	tabsize=4,	
	breaklines=true
}

\DeclareCaptionLabelSeparator{line}{\ --\ }
\DeclareCaptionFont{white}{\color{white}}
\DeclareCaptionFormat{listing}{\colorbox[cmyk]{0.43,0.35,0.35,0.01}{\parbox{\textwidth}{\hspace{15pt}#1#2#3}}}
\captionsetup[lstlisting]{
	singlelinecheck=false,
	labelsep=line
}

\begin{document}
\begin{titlepage}
	\newgeometry{pdftex, left=2cm, right=2cm, top=2.5cm, bottom=2.5cm}
	\fontsize{12pt}{12pt}\selectfont
	\noindent\begin{tabular}{|c|c|}	\hline
	\noindent\begin{minipage}{0.15\textwidth}
		\includegraphics[width=\linewidth]{tools/logo.png}
	\end{minipage} &
	\noindent\begin{minipage}{0.85\textwidth}\centering
		\textbf{\newline Министерство науки и высшего образования Российской Федерации}\\
		\textbf{Федеральное государственное бюджетное образовательное учреждение высшего образования}\\
		\textbf{«Московский государственный технический университет имени Н.Э.~Баумана}\\
		\textbf{(национальный исследовательский университет)»}\\
		\textbf{(МГТУ им. Н.Э.~Баумана)}
	\end{minipage} \\
	\hline	\end{tabular}\newline\newline\newline
	\noindent ФАКУЛЬТЕТ \underline{«Информатика и системы управления»} \newline\newline
	\noindent КАФЕДРА \underline{«Программное обеспечение ЭВМ и информационные технологии»}\newline\newline\newline\newline\newline\newline

	\noindent\begin{minipage}{1.0\textwidth}\centering
		\Large\textbf{   ~~~ Лабораторная работа №4}\newline
		\textbf{по дисциплине «Анализ алгоритмов»}\newline\newline\newline\newline\newline
	\end{minipage}

	\noindent\textbf{Тема} \underline{Параллельные вычисления на основе нативных потоков}\newline\newline
	\textbf{Студент} \underline{Тузов Даниил Александрович}\newline\newline
	\textbf{Группа} \underline{ИУ7-52Б}\newline\newline
	\textbf{Преподаватель} \underline{Строганов Дмитрий Владимирович}
	
	\begin{center}
		\vfill
		Москва, \the\year ~г.
	\end{center}
	\restoregeometry
	\clearpage
\end{titlepage}

\renewcommand{\contentsname}{\begin{center}СОДЕРЖАНИЕ\end{center}} 
\tableofcontents
\setcounter{page}{2}
\clearpage

\begin{center}\section*{ВВЕДЕНИЕ}\end{center}
\addcontentsline{toc}{section}{ВВЕДЕНИЕ}


\clearpage\section{Входные и выходные данные}


\clearpage\section{Преобразование входных данных в выходные}


\clearpage\section{Тестирование}


\clearpage\section{Примеры работы программы}


\clearpage\section{Описание исследования}


\clearpage\begin{center}\section*{ЗАКЛЮЧЕНИЕ}\end{center}
\addcontentsline{toc}{section}{ЗАКЛЮЧЕНИЕ}

\clearpage\begin{center}\section*{\normalsizeСПИСОК ИСПОЛЬЗОВАННЫХ ИСТОЧНИКОВ}\end{center}
\addcontentsline{toc}{section}{СПИСОК ИСПОЛЬЗОВАННЫХ ИСТОЧНИКОВ}
\begin{enumerate}
	\item Никлаус Вирт Алгоритмы и структуры данных. Новая версия для Оберона. Никлаус Вирт. [Текст] / М.: ДМК Пресс, 2016 
— 272 c.
	\item Welcome to Python / [Электронный ресурс] // Режим доступа: https://\\www.python.org
	\item	Библиотека random / [Электронный ресурс] // Режим доступа: https://\\docs.python.org/3/library/random.html
	\item	Библиотека matplotlib / [Электронный ресурс] // Режим доступа: https://\\matplotlib.org
	\item Утилита coverage / [Электронный ресурс] // Режим доступа: https://\\coverage.readthedocs.io/en/latest/
\end{enumerate}

\end{document}